%---------- Inleiding ---------------------------------------------------------

% TODO: Is dit voorstel gebaseerd op een paper van Research Methods die je
% vorig jaar hebt ingediend? Heb je daarbij eventueel samengewerkt met een
% andere student?
% Zo ja, haal dan de tekst hieronder uit commentaar en pas aan.

%\paragraph{Opmerking}

% Dit voorstel is gebaseerd op het onderzoeksvoorstel dat werd geschreven in het
% kader van het vak Research Methods dat ik (vorig/dit) academiejaar heb
% uitgewerkt (met medesturent VOORNAAM NAAM als mede-auteur).
% 

\section{Inleiding}%
\label{sec:inleiding}

In de bedrijfswereld van vandaag staat connectiviteit en beschikbaarheid centraal. Binnen deze context kan een kleine configuratiefout al snel grote gevolgen met zich mee brengen. De manuele configuratie van meerdere netwerktoestellen is een foutgevoelig en tijdsrovend proces, het kan namelijk uren duren om een geheel netwerk te configureren en elke configuratielijn is een nieuwe kans op een fout.
Er is daarom nood aan een manier om het configuratieproces zowel tijdsbesparender te maken als een manier die alle toestellen consistent en foutloos kan configureren, een gepaste oplossing zou automatisatie zijn.
Dit onderzoek zal deze uitdaging proberen aanpakken door het configuratieproces te automatiseren aan de hand van Ansible. Ansible stelt ons in staat een single source of truth (SSoT) te hebben in de vorm van de Ansible controller, wat ervoor zorgt dat alle configuraties centraal beheerd, consistent toegepast en eenvoudig reproduceerbaar zijn over alle netwerktoestellen.
Naast de voordelen van een SSoT, helpt Ansible ook met het hele configuratieproces te versnellen door het een, deels, hands-off proces te maken.
Het doel van deze proof-of-concept is bewijzen dat Ansible een gepaste manier is om netwerkconfiguratie sneller, consistenter, reproduceerbaar en schaalbaar te maken.
Dit onderzoek is gericht op KMO's die vaker beschikken over ICT afdelingen met een beperkt aantal netwerkbeheerders die vaak beperkt zijn in hun tijd om een geheel netwerk toestel per toestel te (her)configureren.

%---------- Stand van zaken ---------------------------------------------------

\section{Literatuurstudie}%
\label{sec:literatuurstudie}

ICT de ruggengraat van het bedrijfsleven noemen is geen toekomstvisie meer, maar de hedendaagse realiteit \autocite{OlalereAbiodun2023}. 
Zo goed als elke organisatie vertrouwt op digitale infrastructuur om hun communicatie en dienstverlening efficiënt te laten verlopen.
Een netwerk dat 24/7 beschikbaar is staat hierin centraal, maar wat als er tijdens de configuratie van het netwerk een fout wordt gemaakt die ervoor zorgt dat cruciale diensten offline worden gehaald?
Om bedrijfscontinuïteit te garanderen is er dus nood aan een manier om het netwerk consistent en foutloos te kunnen configureren.
Waarom automatiseren? Netwerken zijn doorheen de jaren een stuk complexer geworden. Waar vroeger een kleine hoeveelheid toestellen lokaal beheerd werden, beschikken organisaties vandaag over tientallen of honderden netwerktoestellen op verschillende locaties. 
Tientallen toestellen manueel configureren is niet alleen een tijdsrovend proces, een reeks netwerktoestellen productieklaar maken kan al snel uren duren en het proces automatiseren kan dit verkorten naar slechts enkele minuten \autocite{Younes2024}.
Naast de tijd die verloren wordt bij manuele configuratie is er nog een groot probleem: elke configuratielijn is een nieuwe kans op een menselijke fout. 
Automatisatie is een mogelijke oplossing die al getoond heeft dat het aantal menselijke fouten die gemaakt worden aanzienlijk daalt \autocite{Diekmann_2015}.
Hoewel de voordelen van automatisatie duidelijk en gekend zijn, blijven organisaties vaak vasthouden aan manuele configuratie. 
De terughoudendheid van organisaties ligt voornamelijk bij een gebrek aan kennis over automatisatietools, angst voor verandering of de investering in tijd en middelen die nodig is om automatisatie te implementeren \autocite{EMAItential2021}.
Een mogelijke manier om deze zorgen te overbruggen is een gebruiksvriendelijke, goed ondersteunde automatisatietool: Ansible.
Ansible is een open-source automatisatietool, ontwikkeld door Red Hat, die gebruik maakt van een agentless architectuur.
Dit betekent dat er geen aparte software op de netwerktoestellen hoeft te worden geïnstalleerd aangezien Ansible rechtstreeks met de netwerkapparatuur communiceert aan de hand van SSH of API’s. Configuraties worden beschreven in eenvoudige YAML-bestanden, playbooks, waarin de gewenste toestand van het netwerk wordt gedefinieerd.
Deze declaratieve aanpak maakt het mogelijk om complexe configuraties herhaalbaar, consistent en foutloos uit te voeren, waarbij de Ansible controller met alle configuratiebestanden fungeert als een single source of truth (SSoT).
Ansible biedt ook ondersteuning voor een breed assortiment aan netwerkapparatuur, waaronder Cisco waar dit onderzoek zich op gaat focussen, via Ansible Collections.
Hierdoor kunnen KMO’s hun bestaande infrastructuur automatiseren zonder afhankelijk te zijn van één specifieke leverancier, wat bijdraagt aan flexibiliteit en schaalbaarheid \autocite{Cisco2023}.
Naast de eerder genoemde voordelen biedt Ansible ook uitgebreide documentatie en een actieve community, wat het leerproces en de implementatie vergemakkelijkt alsook ondersteuning biedt bij implementatieproblemen.
Dit onderzoek zal ingaan op hoe Ansible gebruikt kan worden in de configuratie van een netwerk binnen KMO's om menselijke fouten te verminderen, het configuratieproces te versnellen en de consistentie, reproduceerbaarheid en schaalbaarheid te vergroten.

%---------- Methodologie ------------------------------------------------------
\section{Methodologie}%
\label{sec:methodologie}

Hier beschrijf je hoe je van plan bent het onderzoek te voeren. Welke onderzoekstechniek ga je toepassen om elk van je onderzoeksvragen te beantwoorden? Gebruik je hiervoor literatuurstudie, interviews met belanghebbenden (bv.~voor requirements-analyse), experimenten, simulaties, vergelijkende studie, risico-analyse, PoC, \ldots?

Valt je onderwerp onder één van de typische soorten bachelorproeven die besproken zijn in de lessen Research Methods (bv.\ vergelijkende studie of risico-analyse)? Zorg er dan ook voor dat we duidelijk de verschillende stappen terug vinden die we verwachten in dit soort onderzoek!

Vermijd onderzoekstechnieken die geen objectieve, meetbare resultaten kunnen opleveren. Enquêtes, bijvoorbeeld, zijn voor een bachelorproef informatica meestal \textbf{niet geschikt}. De antwoorden zijn eerder meningen dan feiten en in de praktijk blijkt het ook bijzonder moeilijk om voldoende respondenten te vinden. Studenten die een enquête willen voeren, hebben meestal ook geen goede definitie van de populatie, waardoor ook niet kan aangetoond worden dat eventuele resultaten representatief zijn.

Uit dit onderdeel moet duidelijk naar voor komen dat je bachelorproef ook technisch voldoen\-de diepgang zal bevatten. Het zou niet kloppen als een bachelorproef informatica ook door bv.\ een student marketing zou kunnen uitgevoerd worden.

Je beschrijft ook al welke tools (hardware, software, diensten, \ldots) je denkt hiervoor te gebruiken of te ontwikkelen.

Probeer ook een tijdschatting te maken. Hoe lang zal je met elke fase van je onderzoek bezig zijn en wat zijn de concrete \emph{deliverables} in elke fase?

%---------- Verwachte resultaten ----------------------------------------------
\section{Verwacht resultaat, conclusie}%
\label{sec:verwachte_resultaten}

Hier beschrijf je welke resultaten je verwacht. Als je metingen en simulaties uitvoert, kan je hier al mock-ups maken van de grafieken samen met de verwachte conclusies. Benoem zeker al je assen en de onderdelen van de grafiek die je gaat gebruiken. Dit zorgt ervoor dat je concreet weet welk soort data je moet verzamelen en hoe je die moet meten.

Wat heeft de doelgroep van je onderzoek aan het resultaat? Op welke manier zorgt jouw bachelorproef voor een meerwaarde?

Hier beschrijf je wat je verwacht uit je onderzoek, met de motivatie waarom. Het is \textbf{niet} erg indien uit je onderzoek andere resultaten en conclusies vloeien dan dat je hier beschrijft: het is dan juist interessant om te onderzoeken waarom jouw hypothesen niet overeenkomen met de resultaten.

