%==============================================================================
% Sjabloon onderzoeksvoorstel bachproef
%==============================================================================
% Gebaseerd op document class `hogent-article'
% zie <https://github.com/HoGentTIN/latex-hogent-article>

% Voor een voorstel in het Engels: voeg de documentclass-optie [english] toe.
% Let op: kan enkel na toestemming van de bachelorproefcoördinator!
\documentclass{hogent-article}

% Invoegen bibliografiebestand
\addbibresource{voorstel.bib}

% Informatie over de opleiding, het vak en soort opdracht
\studyprogramme{Professionele bachelor toegepaste informatica}
\course{Bachelorproef}
\assignmenttype{Onderzoeksvoorstel}
% Voor een voorstel in het Engels, haal de volgende 3 regels uit commentaar
% \studyprogramme{Bachelor of applied information technology}
% \course{Bachelor thesis}
% \assignmenttype{Research proposal}

\academicyear{2025-2026} % TODO: pas het academiejaar aan

% TODO: Werktitel
\title{Automatisatie van netwerkconfiguratie met Ansible: een proof-of-concept voor snellere, consistente en schaalbare configuratie}

% TODO: Studentnaam en emailadres invullen
\author{Edgar Avtandilov}
\email{edgar.avtandilov@student.hogent.be}

% TODO: Medestudent
% Gaat het om een bachelorproef in samenwerking met een student in een andere
% opleiding? Geef dan de naam en emailadres hier
% \author{Yasmine Alaoui (naam opleiding)}
% \email{yasmine.alaoui@student.hogent.be}

% TODO: Geef de co-promotor op
\supervisor[Co-promotor]{Alex Reintjes (Robovision NV, \href{mailto:alexreintjes@hotmail.com}{alexreintjes@hotmail.com})}

% Binnen welke specialisatierichting uit 3TI situeert dit onderzoek zich?
% Kies uit deze lijst:
%
% - Mobile \& Enterprise development
% - AI \& Data Engineering
% - Functional \& Business Analysis
% - System \& Network Administrator
% - Mainframe Expert
% - Als het onderzoek niet past binnen een van deze domeinen specifieer je deze
%   zelf
%
\specialisation{System \& Network Administrator}
\keywords{Automation, Network configuration, Ansible}

\begin{document}

\begin{abstract}
  In de bedrijfswereld staat connectiviteit en beschikbaarheid vandaag de dag centraal, manuele configuratie van netwerktoestellen is nog steeds een courante praktijk en vormt een grote uitdaging voor kmo's. 
  De manuele configuratie van netwerktoestellen is niet alleen een tijdrovend proces, maar ook foutgevoelig aangezien elk commando een nieuwe kans biedt op een menselijke fout. 
  Dit onderzoek richt zich op netwerkconfiguraties automatiseren met Ansible, een open-source automatisatietool die in dit onderzoek als single source of truth (SSoT) wordt gebruikt om configuraties centraal, consistent en reproduceerbaar te maken.
  Door middel van een proof-of-concept in een virtuele kmo-netwerkomgeving wordt onderzocht hoe Ansible kan helpen met snellere, consistentere, schaalbare en foutloze netwerkconfiguratie. 
  Het onderzoek verloopt in vier fasen: voorbereiding, ontwerp, implementatie en evaluatie. 
  Tijdens de implementatie worden virtuele Cisco-netwerktoestellen, IOSv op GNS3, geconfigureerd aan de hand van Ansible.
  Vervolgens worden de resultaten geëvalueerd en wordt er gekeken hoe de vraag \textit{in welke mate kan Ansible bijdragen tot een snellere, reproduceerbare en consistentere, schaalbare en foutloze netwerkconfiguratie binnen kmo's?} kan worden beantwoord.
  De verwachte resultaten tonen aan dat deze manier van automatiseren de configuratietijd met 60-80\% vermindert, de consistentie verhoogt en het aantal menselijke fouten zo goed als elimineert.
  Bovendien blijft de uitvoeringstijd grotendeels constant, ook bij een groeiend aantal toestellen, wat de schaalbaarheid van de oplossing aantoont.
  Tot slot, Ansible biedt een laagdrempelige, kostenefficiënte en betrouwbare manier voor kmo's die hun netwerkbeheer willen optimaliseren aan de hand van automatisatie, zonder nood aan complexe infrastructuuraanpassingen, lange uren voor hun netwerkbeheerders of dure, gespecialiseerde software.
\end{abstract}

\tableofcontents

% De hoofdtekst van het voorstel zit in een apart bestand, zodat het makkelijk
% kan opgenomen worden in de bijlagen van de bachelorproef zelf.
%---------- Inleiding ---------------------------------------------------------

% TODO: Is dit voorstel gebaseerd op een paper van Research Methods die je
% vorig jaar hebt ingediend? Heb je daarbij eventueel samengewerkt met een
% andere student?
% Zo ja, haal dan de tekst hieronder uit commentaar en pas aan.

%\paragraph{Opmerking}

% Dit voorstel is gebaseerd op het onderzoeksvoorstel dat werd geschreven in het
% kader van het vak Research Methods dat ik (vorig/dit) academiejaar heb
% uitgewerkt (met medesturent VOORNAAM NAAM als mede-auteur).
% 

\section{Inleiding}%
\label{sec:inleiding}

In de bedrijfswereld van vandaag staat connectiviteit en beschikbaarheid centraal. Binnen deze context kan een kleine configuratiefout al snel grote gevolgen met zich mee brengen. De manuele configuratie van meerdere netwerktoestellen is een foutgevoelig en tijdrovend proces, het kan namelijk uren duren om een geheel netwerk te configureren en elke configuratielijn is een nieuwe kans op een fout.
Er is daarom nood aan een manier om het configuratieproces zowel tijdsbesparender te maken als een manier die alle toestellen consistent en foutloos kan configureren, een gepaste oplossing zou automatisatie zijn.
Dit onderzoek zal deze uitdaging aanpakken door het configuratieproces te automatiseren aan de hand van Ansible. Ansible stelt ons in staat een single source of truth (SSoT) te hebben in de vorm van de Ansible controller, wat ervoor zorgt dat alle configuraties centraal beheerd, consistent toegepast en eenvoudig reproduceerbaar zijn over alle netwerktoestellen.
Naast de voordelen van een SSoT, helpt Ansible ook het hele configuratieproces te versnellen door het een, deels, hands-off proces te maken.
Deze elementen leiden naar de onderzoeksvraag \textit{in welke mate kan Ansible bijdragen aan een snellere, reproduceerbare en consistentere, schaalbare en foutloze netwerkconfiguratie binnen kmo's?}.
Om deze vraag te beantwoorden zijn de volgende deelvragen geformuleerd, \textit{hoeveel sneller is een netwerkconfiguratie aan de hand van ansible?}, \textit{blijft configuratie consistent bij het heruitvoeren van de Ansible playbook?}, \textit{is configuratie aan de hand van Ansible even snel bij een klein netwerk als bij een middelgroot netwerk?}, \textit{worden er minder fouten die connectiviteit verbreken gemaakt bij een geautomatiseerde configuratie vergeleken met de traditionele manier van configureren?}.
Het doel van deze proof-of-concept is bewijzen dat Ansible een gepaste manier is om netwerkconfiguratie sneller, consistenter, reproduceerbaar en schaalbaar te maken.
Dit onderzoek is gericht op kmo's die vaker beschikken over ICT-afdelingen met een beperkt aantal netwerkbeheerders die vaak beperkt zijn in hun tijd om een geheel netwerk, toestel per toestel, te (her)configureren.

%---------- Stand van zaken ---------------------------------------------------

\section{Literatuurstudie}%
\label{sec:literatuurstudie}

ICT de ruggengraat van het bedrijfsleven noemen is geen toekomstvisie meer, maar de hedendaagse realiteit \autocite{OlalereAbiodun2023}. 
Zo goed als elke organisatie vertrouwt op digitale infrastructuur om hun communicatie en dienstverlening efficiënt te laten verlopen.
Een netwerk dat 24/7 beschikbaar is staat hierin centraal, maar wat als er tijdens de configuratie van het netwerk een fout wordt gemaakt die ervoor zorgt dat cruciale diensten offline worden gehaald?
Om bedrijfscontinuïteit te garanderen is er dus nood aan een manier om het netwerk consistent en foutloos te kunnen configureren.
Waarom automatiseren? Netwerken zijn doorheen de jaren een stuk complexer geworden. Waar vroeger een kleine hoeveelheid toestellen lokaal beheerd werden, beschikken organisaties vandaag over tientallen of honderden netwerktoestellen op verschillende locaties. 
Tientallen toestellen manueel configureren is niet alleen een tijdrovend proces, een reeks netwerktoestellen productieklaar maken kan al snel uren duren en het proces automatiseren kan dit verkorten naar slechts enkele minuten \autocite{Younes2024}.
Naast de tijd die verloren wordt bij manuele configuratie is er nog een groot probleem: elke configuratielijn is een nieuwe kans op een menselijke fout. 
Automatisatie is een mogelijke oplossing die al getoond heeft dat het aantal menselijke fouten die gemaakt worden aanzienlijk daalt \autocite{Diekmann_2015}.
Hoewel de voordelen van automatisatie duidelijk en gekend zijn, blijven organisaties vaak vasthouden aan manuele configuratie. 
De terughoudendheid van organisaties ligt voornamelijk bij een gebrek aan kennis over automatisatietools, angst voor verandering of de investering in tijd en middelen die nodig is om automatisatie te implementeren \autocite{EMAItential2021}.
Een mogelijke manier om deze zorgen te overbruggen is een gebruiksvriendelijke, goed ondersteunde automatisatietool: Ansible.
Ansible is een open-source automatisatietool, ontwikkeld door Red Hat, die gebruik maakt van een agentless architectuur.
Dit betekent dat er geen aparte software op de netwerktoestellen hoeft te worden geïnstalleerd aangezien Ansible rechtstreeks met de netwerkapparatuur communiceert aan de hand van SSH of API's. Configuraties worden beschreven in eenvoudige YAML-bestanden, playbooks, waarin de gewenste toestand van het netwerk wordt gedefinieerd.
Deze declaratieve aanpak maakt het mogelijk om complexe configuraties herhaalbaar, consistent en foutloos uit te voeren, waarbij de Ansible controller (een server of computer vanwaar alle Ansible-taken worden uitgevoerd en beheerd) met alle configuratiebestanden fungeert als een single source of truth (SSoT).
Ansible biedt ook ondersteuning voor een breed assortiment aan netwerkapparatuur, waaronder Cisco, waarop dit onderzoek zich gaat focussen, via Ansible Collections.
Hierdoor kunnen kmo's hun bestaande infrastructuur automatiseren zonder afhankelijk te zijn van één specifieke leverancier, wat bijdraagt aan flexibiliteit en schaalbaarheid \autocite{Cisco2022}.
Naast de eerder genoemde voordelen biedt Ansible ook uitgebreide documentatie en een actieve community, wat het leerproces en de implementatie vergemakkelijkt alsook ondersteuning biedt bij implementatieproblemen.
Dit onderzoek zal ingaan op hoe Ansible gebruikt kan worden in de configuratie van een netwerk binnen kmo's om menselijke fouten te verminderen, het configuratieproces te versnellen en de consistentie, reproduceerbaarheid en schaalbaarheid te vergroten.

%---------- Methodologie ------------------------------------------------------
\section{Methodologie}%
\label{sec:methodologie}

Om de proof-of-concept te realiseren, wordt een kwantitatieve en experimentele onderzoeksmethode toegepast. Deze methodologie bestaat uit vier hoofdfasen: voorbereiding, ontwerp, implementatie en evaluatie.

\subsection{Voorbereidingsfase}
In de voorbereidingsfase wordt de bestaande netwerkarchitectuur van een typische kmo bestudeerd om een realistische testomgeving te kunnen opzetten. Het einddoel van deze fase is het volgende in kaart brengen:
\begin{itemize}
  \item Welke netwerktoestellen (zoals switches en routers) zullen gebruikt worden.
  \item Welke configuratieparameters representatief zijn voor een netwerktoestel binnen een kmo (zoals VLAN's, IP-adressering en routing).
  \item Welke handelingen netwerkbeheerders typisch manueel uitvoeren.
\end{itemize}
Op basis van deze analyse wordt een representatieve testomgeving uitgewerkt die de netwerkstructuur van een kmo nabootst. Dit gebeurt in een virtuele omgeving om risico's op productieomgevingen te vermijden.

\subsection{Ontwerpfase}
In deze fase wordt het automatisatieconcept ontworpen aan de hand van Ansible. Hierbij worden:
\begin{itemize}
  \item Een Ansible controller opgezet als centraal beheerpunt.
  \item Playbooks opgesteld die de gewenste netwerkconfiguratie beschrijven in YAML-formaat.
  \item Inventarissen aangemaakt die de netwerktoestellen definiëren en groeperen.
\end{itemize}
Daarnaast wordt het concept van een single source of truth (SSoT) geïntegreerd door alle configuratiebestanden centraal te beheren op de controller. Dit ontwerp garandeert consistentie en herhaalbaarheid over alle toestellen.

\subsection{Implementatiefase}
De implementatiefase richt zich op het toepassen en testen van de Ansible-configuraties. De volgende stappen worden uitgevoerd:
\begin{enumerate}
  \item Configuratie van virtuele netwerktoestellen (Cisco IOS in GNS3).
  \item Verbinding tussen de Ansible controller en de netwerktoestellen via SSH.
  \item Uitvoering van playbooks voor basisconfiguratie (hostname, interfaces, VLAN's, routing, etc).
  \item Herhaalde uitvoering van dezelfde playbooks om de reproduceerbaarheid te testen.
\end{enumerate}
Tijdens deze fase wordt gemeten hoeveel tijd nodig is voor configuratie met en zonder automatisatie, en wordt het aantal fouten geregistreerd om de impact van automatisatie te evalueren.

\subsection{Evaluatiefase}
Tot slot wordt het proof-of-concept geëvalueerd op basis van meetbare criteria:
\begin{itemize}
  \item \textbf{Tijdswinst}: vergelijking van manuele versus geautomatiseerde configuratie.
  \item \textbf{Reproduceerbaarheid en Consistentie}: mogelijkheid om dezelfde configuratie meerdere keren zonder afwijkingen toe te passen.
  \item \textbf{Schaalbaarheid}: evaluatie van de prestaties bij meerdere netwerktoestellen.
  \item \textbf{Foutvermindering}: Worden minder fouten gemaakt bij een configuratie met behulp van Ansible.

\end{itemize}
De resultaten van deze evaluatie dienen om te bepalen of Ansible effectief een haalbare en efficiënte oplossing biedt voor netwerkconfiguratie binnen kmo's.

\subsection{Tools en technologieën}
Voor dit onderzoek worden de volgende tools en technologieën gebruikt:
\begin{itemize}
  \item \textbf{Ansible}: automatisatietool voor configuratiebeheer.
  \item \textbf{Cisco IOS} (virtueel): voor representatieve netwerktoestellen.
  \item \textbf{GNS3}: simulatieplatform voor netwerkarchitecturen.
  \item \textbf{Python en YAML}: voor het schrijven en aanpassen van playbooks.
\end{itemize}
Deze combinatie van technologieën maakt het mogelijk om het volledige configuratieproces gecontroleerd, herhaalbaar en meetbaar uit te voeren.


%---------- Verwachte resultaten ----------------------------------------------
\section{Verwacht resultaat, conclusie}%
\label{sec:verwachte_resultaten}

Het doel van dit onderzoek is om aan te tonen dat netwerkconfiguratie automatiseren met Ansible leidt tot een snellere, consistentere en meer schaalbare manier van configureren waarbij minder fouten worden gemaakt in vergelijking met traditionele, manuele configuratie.
De verwachte resultaten worden opgesplitst in vier categorieën: tijdswinst, reproduceerbaarheid en consistentie, schaalbaarheid en foutvermindering.

\subsection{Tijdswinst}

Ansible playbooks stellen netwerkbeheerders in staat de configuratietijd aanzienlijk te verminderen.
Een netwerk configureren op de traditionele manier houdt in dat de netwerkbeheerder elk commando op elk toestel manueel moet invoeren, Ansible daarentegen voert de configuratie parallel uit op alle toestellen.
Door de configuratie parallel uit te voeren wordt er een tijdwinst van 60-80\% verwacht, afhankelijk van de grootte van het netwerk en de complexiteit van de configuratie.
De grootste tijdswinst zal echter te vinden zijn bij herconfiguratie of aanpassingen, doordat alles centraal beheerd wordt in de inventaris van de controller. 
Aangezien de configuratie, zoals eerder vernoemd, parallel verloopt en op basis van de controller als SSoT, moet een wijziging in adressering, routing of VLAN's niet handmatig op elk toestel worden uitgevoerd.
De netwerkbeheerder moet enkel de gewenste aanpassing doorvoeren in playbooks en inventarisbestanden en de playbooks uitvoeren.
Ansible zorgt er vervolgens voor dat deze wijziging automatisch en consistent wordt toegepast op alle relevante toestellen.

\subsection{Reproduceerbaarheid en Consistentie}

Netwerkconfiguratie consistent houden is nog een voordeel dat Ansible met zich meebrengt. 
De declaratieve aanpak houdt in dat alle toestellen en configuraties in inventarisbestanden en playbooks worden verwerkt.
Deze aanpak brengt het voordeel van consistentie met zich mee aangezien alles op dezelfde manier wordt geconfigureerd.
Ook worden configuraties die afhankelijk zijn van verwijzingen altijd meegenomen vanuit de inventarisbestanden en playbooks.
Deze voordelen zijn vooral merkbaar bij aanpassingen of uitbreidingen van het netwerk.  
Zo kan bijvoorbeeld een routingtabel eenvoudig worden aangepast of aangevuld op basis van de inventaris, zonder dat er nog verwijzingen bestaan naar servers die een nieuw IP-adres hebben gekregen 
of hoeft bij de toevoeging van een nieuwe router niet langer op elke bestaande router handmatig een extra route worden geconfigureerd — Ansible past de wijziging automatisch toe op alle relevante toestellen.

\subsection{Schaalbaarheid}
Naarmate het aantal netwerktoestellen toeneemt, blijft de uitvoeringstijd grotendeels constant, aangezien de configuratie parallel kan worden uitgerold.  
Het verwachte resultaat is dat de complexiteit van het netwerk slechts een beperkte invloed heeft op de configuratietijd, wat aantoont dat de oplossing schaalbaar is naar grotere bedrijfsnetwerken.

\subsection{Foutvermindering}
Omdat alle configuraties in één centraal beheerd bestand (de Ansible controller) worden opgeslagen en uitgerold, wordt verwacht dat het aantal menselijke fouten sterk afneemt.  
Configuraties die met Ansible worden uitgerold zijn bovendien herhaalbaar en identiek, wat menselijke typfouten of inconsistenties uitsluit.  
Het verwachte resultaat is een daling van het aantal configuratiefouten tot nagenoeg nul, aangezien elke wijziging via hetzelfde playbook wordt toegepast op alle toestellen.

\subsection{Conclusie}
Ansible is een makkelijke en goedkope manier om netwerkconfiguratie sneller, reproduceerbaar, consistenter en meer schaalbaar te maken, alsook het aantal fouten dat gemaakt wordt bij de configuratie drastisch te verminderen.
Het is een mature technologie met veel documentatie en een grote community wat ervoor zorgt dat het schrijven van playbooks en de implementatie ervan een laagdrempelige stap naar automatisatie is.


\printbibliography[heading=bibintoc]

\end{document}