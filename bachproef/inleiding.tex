%%=============================================================================
%% Inleiding
%%=============================================================================

\chapter{\IfLanguageName{dutch}{Inleiding}{Introduction}}%
\label{ch:inleiding}

De inleiding moet de lezer net genoeg informatie verschaffen om het onderwerp te begrijpen en in te zien waarom de onderzoeksvraag de moeite waard is om te onderzoeken. In de inleiding ga je literatuurverwijzingen beperken, zodat de tekst vlot leesbaar blijft. Je kan de inleiding verder onderverdelen in secties als dit de tekst verduidelijkt. Zaken die aan bod kunnen komen in de inleiding~\autocite{Pollefliet2011}:

\begin{itemize}
  \item context, achtergrond
  \item afbakenen van het onderwerp
  \item verantwoording van het onderwerp, methodologie
  \item probleemstelling
  \item onderzoeksdoelstelling
  \item onderzoeksvraag
  \item \ldots
\end{itemize}

\section{\IfLanguageName{dutch}{Probleemstelling}{Problem Statement}}%
\label{sec:probleemstelling}

Uit je probleemstelling moet duidelijk zijn dat je onderzoek een meerwaarde heeft voor een concrete doelgroep. De doelgroep moet goed gedefinieerd en afgelijnd zijn. Doelgroepen als ``bedrijven,'' ``KMO's'', systeembeheerders, enz.~zijn nog te vaag. Als je een lijstje kan maken van de personen/organisaties die een meerwaarde zullen vinden in deze bachelorproef (dit is eigenlijk je steekproefkader), dan is dat een indicatie dat de doelgroep goed gedefinieerd is. Dit kan een enkel bedrijf zijn of zelfs één persoon (je co-promotor/opdrachtgever).

\section{\IfLanguageName{dutch}{Onderzoeksvraag}{Research question}}%
\label{sec:onderzoeksvraag}

In dit onderzoek wordt onderzocht in welke mate Ansible gebruikt kan worden om netwerkconfiguratie sneller, reproduceerbaar en consistent, schaalbaar en foutloos te maken.
Om deze vraag specifiek te kunnen beantwoorden wordt gekeken naar de volgende deelvragen:
\begin{itemize}
  \item Hoeveel sneller is een geautomatiseerde configuratie in vergelijking met een manuele configuratie van het netwerk?

  \item Blijft de configuratie van de netwerktoestellen consistent bij het heruitvoeren van de Ansible-playbooks?

  \item Is configuratie aan de hand van Ansible even snel bij een klein netwerk als bij een middelgroot netwerk?
  
  \item Worden minder fouten die connectiviteit verbreken gemaakt bij een geautomatiseerde configuratie vergeleken met de traditionele manier van configureren?
\end{itemize}

\section{\IfLanguageName{dutch}{Onderzoeksdoelstelling}{Research objective}}%
\label{sec:onderzoeksdoelstelling}

Deze bachelorproef heeft als doel een proof-of-concept realiseren dat de meerwaarde van netwerkautomatisatie met Ansible binnen een bedrijfscontext aantoont.
Concreet wordt een netwerktopologie opgezet op basis van een gemiddeld kmo-netwerk in GNS3. Het netwerk bestaat uit een centrale bedrijfsrouter en 3 sites, elk met een router en meerdere switches, waarop configuratie- en beheerhandelingen geautomatiseerd worden uitgevoerd via Ansible-playbooks.

Het project is succesvol wanneer:
\begin{itemize}
  \item het gehele netwerk geautomatiseerd configureren sneller is dan het netwerk manueel te configureren ;
  
  \item de Ansible-playbooks foutloos configuraties kunnen toepassen op alle netwerkapparaten ;

  \item de configuraties consistent en reproduceerbaar zijn;

  \item de beheerder taken zoals interface-configuratie, routing en NAT kan uitvoeren zonder manuele CLI-invoer;

  \item fouten die connectiviteit onderbreken .
\end{itemize}

\section{\IfLanguageName{dutch}{Opzet van deze bachelorproef}{Structure of this bachelor thesis}}%
\label{sec:opzet-bachelorproef}

% Het is gebruikelijk aan het einde van de inleiding een overzicht te
% geven van de opbouw van de rest van de tekst. Deze sectie bevat al een aanzet
% die je kan aanvullen/aanpassen in functie van je eigen tekst.

De rest van deze bachelorproef is als volgt opgebouwd:

In Hoofdstuk~\ref{ch:stand-van-zaken} wordt een overzicht gegeven van de stand van zaken binnen het onderzoeksdomein, op basis van een literatuurstudie.

In Hoofdstuk~\ref{ch:methodologie} wordt de methodologie toegelicht en worden de gebruikte onderzoekstechnieken besproken om een antwoord te kunnen formuleren op de onderzoeksvragen.

% TODO: Vul hier aan voor je eigen hoofstukken, één of twee zinnen per hoofdstuk

In Hoofdstuk~\ref{ch:conclusie}, tenslotte, wordt de conclusie gegeven en een antwoord geformuleerd op de onderzoeksvragen. Daarbij wordt ook een aanzet gegeven voor toekomstig onderzoek binnen dit domein.