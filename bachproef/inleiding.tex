%%=============================================================================
%% Inleiding
%%=============================================================================

\chapter{\IfLanguageName{dutch}{Inleiding}{Introduction}}%
\label{ch:inleiding}

In moderne bedrijfsnetwerken staat connectiviteit, beschikbaarheid en consistentie centraal. 
Naarmate ICT-infrastructuur groeit en complexer wordt, neemt ook de beheerslast toe. 
Netwerkbeheerders worden geconfronteerd met een toenemende hoeveelheid configuratiewijzigingen die manueel in de command line interface (CLI) uitgevoerd moeten worden op routers, switches en andere netwerkapparaten. 
In de CLI werken is niet alleen tijdrovend, maar ook foutgevoelig. 
Een klein typefoutje kan immers leiden tot downtime of inconsistent gedrag binnen het netwerk. 

Om deze uitdagingen aan te pakken, kijken bedrijven steeds meer naar automatisatie. 
Door configuraties te centraliseren en automatisch uit te rollen, kunnen organisaties tijd besparen, menselijke fouten verminderen en hun infrastructuur sneller aanpassen aan veranderende bedrijfsbehoeften. 
Een van de tools die hiervoor ingezet kan worden, is \textbf{Ansible}, een open-source automatisatietool die netwerkbeheerders in staat stelt configuraties via eenvoudige YAML-scripts te beheren. 
Deze bachelorproef onderzoekt in welke mate Ansible gebruikt kan worden om netwerkconfiguratie te automatiseren en welke voordelen dit met zich mee brengt.

\subsection{Afbakening van het onderwerp}


Deze bachelorproef focust op de automatisering van de configuratie op Cisco IOS netwerktoestellen binnen een gesimuleerde bedrijfsomgeving, opgebouwd in GNS3.
De nadruk ligt op het toepassen van Ansible voor het configureren van Cisco routers en switches.


\subsection{Verantwoording en methodologie}


Het onderwerp is maatschappelijk en technologisch relevant omdat netwerkautomatisering een sleutelrol speelt in het verminderen van operationele kosten en het verhogen van betrouwbaarheid.
Het onderzoek volgt een praktijkgericht karakter: er wordt een testomgeving ontworpen waarin Ansible-playbooks worden ontwikkeld, getest en geëvalueerd. 
De methodologie omvat het ontwerpen van een netwerkarchitectuur, het schrijven van configuratiescripts en het valideren van de resultaten op basis van meetbare criteria zoals tijdbesparing, consistentie en reproduceerbaarheid, schaalbaarheid en foutafhandeling.


\section{\IfLanguageName{dutch}{Probleemstelling}{Problem Statement}}%
\label{sec:probleemstelling}

Manuele configuratie van netwerkapparatuur is foutgevoelig, moeilijk reproduceerbaar en slecht schaalbaar. 
Dit vormt een probleem voor organisaties die streven naar gestandaardiseerde en snel aanpasbare netwerkomgevingen. 
Er is nood aan een geautomatiseerde oplossing die deze tekortkomingen opvangt zonder de betrouwbaarheid van bestaande netwerken te schaden.

\section{\IfLanguageName{dutch}{Onderzoeksvraag}{Research question}}%
\label{sec:onderzoeksvraag}

In dit onderzoek wordt onderzocht in welke mate Ansible gebruikt kan worden om netwerkconfiguratie sneller, reproduceerbaar en consistent, schaalbaar en foutloos te maken.
Om deze vraag specifiek te kunnen beantwoorden wordt gekeken naar de volgende deelvragen:
\begin{itemize}
  \item Hoeveel sneller is een geautomatiseerde configuratie in vergelijking met een manuele configuratie van het netwerk?

  \item Blijft de configuratie van de netwerktoestellen consistent bij het heruitvoeren van de Ansible-playbooks?

  \item Is configuratie aan de hand van Ansible even snel bij een klein netwerk als bij een middelgroot netwerk?
  
  \item Worden minder fouten die connectiviteit verbreken gemaakt bij een geautomatiseerde configuratie vergeleken met de traditionele manier van configureren?
\end{itemize}

\section{\IfLanguageName{dutch}{Onderzoeksdoelstelling}{Research objective}}%
\label{sec:onderzoeksdoelstelling}

Deze bachelorproef heeft als doel een proof-of-concept realiseren dat de meerwaarde van netwerkautomatisatie met Ansible binnen een bedrijfscontext aantoont.
Concreet wordt een netwerktopologie opgezet op basis van een gemiddeld kmo-netwerk in GNS3. Het netwerk bestaat uit een centrale bedrijfsrouter en 3 sites, elk met een router en meerdere switches, waarop configuratie- en beheerhandelingen geautomatiseerd worden uitgevoerd via Ansible-playbooks.

Het project is succesvol wanneer:
\begin{itemize}
  \item het gehele netwerk geautomatiseerd configureren sneller is dan het netwerk manueel te configureren ;
  
  \item de Ansible-playbooks foutloos configuraties kunnen toepassen op alle netwerkapparaten ;

  \item de configuraties consistent en reproduceerbaar zijn;

  \item de beheerder taken zoals interface-configuratie, routing en NAT kan uitvoeren zonder manuele CLI-invoer;

  \item fouten die connectiviteit onderbreken .
\end{itemize}

\section{\IfLanguageName{dutch}{Opzet van deze bachelorproef}{Structure of this bachelor thesis}}%
\label{sec:opzet-bachelorproef}

% Het is gebruikelijk aan het einde van de inleiding een overzicht te
% geven van de opbouw van de rest van de tekst. Deze sectie bevat al een aanzet
% die je kan aanvullen/aanpassen in functie van je eigen tekst.

De rest van deze bachelorproef is als volgt opgebouwd:

In Hoofdstuk~\ref{ch:stand-van-zaken} wordt een overzicht gegeven van de stand van zaken binnen het onderzoeksdomein, op basis van een literatuurstudie.

In Hoofdstuk~\ref{ch:methodologie} wordt de methodologie toegelicht en worden de gebruikte onderzoekstechnieken besproken om een antwoord te kunnen formuleren op de onderzoeksvragen.

% TODO: Vul hier aan voor je eigen hoofstukken, één of twee zinnen per hoofdstuk

In Hoofdstuk~\ref{ch:conclusie}, tenslotte, wordt de conclusie gegeven en een antwoord geformuleerd op de onderzoeksvragen. Daarbij wordt ook een aanzet gegeven voor toekomstig onderzoek binnen dit domein.