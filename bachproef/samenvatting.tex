%%=============================================================================
%% Samenvatting
%%=============================================================================

% TODO: De "abstract" of samenvatting is een kernachtige (~ 1 blz. voor een
% thesis) synthese van het document.
%
% Een goede abstract biedt een kernachtig antwoord op volgende vragen:
%
% 1. Waarover gaat de bachelorproef?
% 2. Waarom heb je er over geschreven?
% 3. Hoe heb je het onderzoek uitgevoerd?
% 4. Wat waren de resultaten? Wat blijkt uit je onderzoek?
% 5. Wat betekenen je resultaten? Wat is de relevantie voor het werkveld?
%
% Daarom bestaat een abstract uit volgende componenten:
%
% - inleiding + kaderen thema
% - probleemstelling
% - (centrale) onderzoeksvraag
% - onderzoeksdoelstelling
% - methodologie
% - resultaten (beperk tot de belangrijkste, relevant voor de onderzoeksvraag)
% - conclusies, aanbevelingen, beperkingen
%
% LET OP! Een samenvatting is GEEN voorwoord!

%%---------- Nederlandse samenvatting -----------------------------------------
%
% TODO: Als je je bachelorproef in het Engels schrijft, moet je eerst een
% Nederlandse samenvatting invoegen. Haal daarvoor onderstaande code uit
% commentaar.
% Wie zijn bachelorproef in het Nederlands schrijft, kan dit negeren, de inhoud
% wordt niet in het document ingevoegd.

\IfLanguageName{english}{%
\selectlanguage{dutch}
\chapter*{Samenvatting}
\lipsum[1-4]
\selectlanguage{english}
}{}

%%---------- Samenvatting -----------------------------------------------------
% De samenvatting in de hoofdtaal van het document

\chapter*{\IfLanguageName{dutch}{Samenvatting}{Abstract}}
ICT de ruggengraat van het bedrijfsleven noemen is geen toekomstvisie meer, maar de realiteit van vandaag, net als netwerktoestellen manueel configureren.
De manuele configuratie van netwerktoestellen vormt een grote uitdaging voor bedrijven. Manuele configuratie is niet alleen traag, maar ook zeer foutgevoelig. Elk nieuw commando dat wordt uitgevoerd is namelijk een nieuwe kans voor de netwerkadministrator om een een kleine, menselijke fout te maken.
Wanneer connectiviteit en beschikbaarheid centraal staan in de bedrijfscontinuïteit, kan een menselijke fout grote gevolgen met zich mee brengen.
Dit onderzoek richt zich op een manier om netwerkconfiguratie te automatiseren aan de hand van Ansible, een open-source automatisatietool. Het gebruik van Ansible stelt ons er toe in staat in dit onderzoek een Ansible controller in te zetten die fungeert als single source of truth (SSoT). 
De combinatie van automatisatie en een SSoT leidt naar de onderzoeksvraag \textit{in welke mate kan Ansible bijdragen aan een snellere, reproduceerbare en consistentere, schaalbare en foutloze netwerkconfiguratie}?
Deze vraag wordt beantwoord in de proof-of-concept aan de hand van 4 fasen. In de voorbereidingsfase wordt de netwerkarchitectuur van een typische kmo bestudeerd om een realistische testomgeving op te zetten.
In de ontwerpfase wordt een Ansible controller opgezet en worden de nodige playbooks en inventarisbestanden opgemaakt. In de implementatiefase wordt het (virtueel) netwerk geconfigureerd aan de hand van de Ansible controller. Tijdens de implementatiefase wordt ook gemeten hoelang het netwerk opzetten duurt en hoeveel fouten er worden gemaakt bij zowel de manuele configuratie als de geautomatiseerde configuratie.
Tot slot worden de metingen ge-evalueerd om volgende vragen te kunnen beantwoorden. \textit{Hoeveel sneller is een netwerkconfiguratie aan de hand van ansible}? 
\textit{Blijft configuratie consistent bij het heruitvoeren van de Ansible playbook}? \textit{Is configuratie aan de hand van Ansible even snel bij een klein netwerk als bij een middelgroot netwerk}?
\textit{Worden minder fouten die connectiviteit verbreken gemaakt bij een geautomatiseerde configuratie vergeleken met de traditionele manier van configureren}?
Dit onderzoek toont aan dat netwerkconfiguratie automatiseren met Ansible in lijn staat met alle verwachtingen. 
De netwerkconfiguratie verloopt (na het schrijven van de nodige playbooks en inventarisbestanden) tot wel 70\% sneller, is consistent en reproduceerbaar en brengt minder fouten met zich mee. De voordelen van de snelheid zijn voornamelijk voelbaar bij herconfiguratie en het schalen van het netwerk. 
Een middelgroot netwerk (16 netwerktoestellen) configureren duurt namelijk maar een beetje langer dan een klein netwerk (5 netwerktoestellen), terwijl de besteede tijd bij een handmatige configuratie lineair meegroeit.
Dit onderzoek concludeerd dat een Ansible een laagdrempelige, kostenefficiënte en betrouwbare manier biedt voor kmo's om hun netwerkbeheer te optimaliseren aan de hand van automatisatie, zonder nood aan complexe infrastuctuuraanpassingen, dure en gespecialiseerde software of lange werkuren voor hun netwerkbeheerders.