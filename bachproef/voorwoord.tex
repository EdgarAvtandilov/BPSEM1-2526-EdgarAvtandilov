%%=============================================================================
%% Voorwoord
%%=============================================================================

\chapter*{\IfLanguageName{dutch}{Woord vooraf}{Preface}}%
\label{ch:voorwoord}

%% TODO:
%% Het voorwoord is het enige deel van de bachelorproef waar je vanuit je
%% eigen standpunt (``ik-vorm'') mag schrijven. Je kan hier bv. motiveren
%% waarom jij het onderwerp wil bespreken.
%% Vergeet ook niet te bedanken wie je geholpen/gesteund/... heeft


Deze bachelorproef vormt het sluitstuk van mijn opleiding en biedt de mogelijkheid om theoretische kennis die ik heb opgedaan tijdens mijn opleiding te combineren met een actueel en praktisch probleem uit het werkveld. 
Tijdens mijn opleiding groeide mijn interesse in netwerken en infrastructuurbeheer, en in het bijzonder in de toenemende complexiteit van moderne netwerkomgevingen. 
Deze vaststelling vormde de aanleiding om netwerkautomatisatie als onderwerp voor deze bachelorproef te kiezen.
In deze bachelorproef wordt onderzocht hoe netwerkautomatisatie, met behulp van Ansible, kan bijdragen aan een snelle, reproduceerbare, consistente, schaalbare en minder foutgevoelige netwerkconfiguratie binnen een kmo-context. 
Het uitwerken van deze bachelorproef bood de kans om zowel theoretische inzichten als praktische vaardigheden te combineren en verder te ontwikkelen.
Graag wil ik mijn promotor, Irina Malfait, bedanken voor de begeleiding en de waardevolle feedback tijdens het uitwerken van deze bachelorproef. 
Daarnaast wil ik ook mijn co-promotor, Alex Reintjes, bedanken voor de technische ondersteuning en inhoudelijke inzichten.
Tot slot wil ik mijn familie en vrienden bedanken voor hun steun en begrip tijdens het schrijven van deze bachelorproef.