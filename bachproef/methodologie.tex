%%=============================================================================
%% Methodologie
%%=============================================================================

\chapter{\IfLanguageName{dutch}{Methodologie}{Methodology}}%
\label{ch:methodologie}

%% TODO: In dit hoofstuk geef je een korte toelichting over hoe je te werk bent
%% gegaan. Verdeel je onderzoek in grote fasen, en licht in elke fase toe wat
%% de doelstelling was, welke deliverables daar uit gekomen zijn, en welke
%% onderzoeksmethoden je daarbij toegepast hebt. Verantwoord waarom je
%% op deze manier te werk gegaan bent.
%% 
%% Voorbeelden van zulke fasen zijn: literatuurstudie, opstellen van een
%% requirements-analyse, opstellen long-list (bij vergelijkende studie),
%% selectie van geschikte tools (bij vergelijkende studie, "short-list"),
%% opzetten testopstelling/PoC, uitvoeren testen en verzamelen
%% van resultaten, analyse van resultaten, ...
%%
%% !!!!! LET OP !!!!!
%%
%% Het is uitdrukkelijk NIET de bedoeling dat je het grootste deel van de corpus
%% van je bachelorproef in dit hoofstuk verwerkt! Dit hoofdstuk is eerder een
%% kort overzicht van je plan van aanpak.
%%
%% Maak voor elke fase (behalve het literatuuronderzoek) een NIEUW HOOFDSTUK aan
%% en geef het een gepaste titel.

\section{Voorbereidingsfase}
In deze fase wordt onderzocht hoe een gemiddelde netwerktopologie van een kmo eruit ziet om een representatieve testomgeving op te zetten.
Specifiek wordt er gekeken naar hoeveel toestellen, welk type toestellen en welke configuratie de toestellen moeten hebben.
Om het type toestellen te bepalen wordt er voornamelijk gekeken naar bestaande literatuur om te zien welk type toestellen het grootste marktaandeel hebben, dit vergroot de representativiteit van de testomgeving en de toepasbaarheid van de resultaten. 
Daarbij wordt ook rekening gehouden met configuratieprofielen en beheeropties, zodat de gekozen apparaten zoveel mogelijk aansluiten bij de praktijk van kmo's.
Deze fase toont dat Cisco, voor zowel switches als routers, het grootste marktaandeel heeft. Daarom werd er gekozen om voor de proof-of-concept Cisco toestellen te gebruiken in de testomgeving.


\section{Ontwerpfase}
In de ontwerpfase worden de testomgeving en Ansible controller effectief opgezet, hiervoor wordt een virtueel netwerk opgezet in GNS3.
Er werd gekozen een bedrijfsomgeving te simuleren met meerdere sites, dit werd gedaan aangezien een kleine topologie geen goed beeld zou geven op de schaalbaarheid en tijdsbesparende kwaliteiten van automatisatie met Ansible.
Meerdere sites geven ook de mogelijkheid representatieve configuratie, zoals routing, Open Shortest Path First (OSPF) en Access Control Lists (ACL) te implementeren. 
In deze fase worden ook het Ansible playbook en inventarisbestanden aangemaakt.
In de inventarisbestanden worden alle routers en switches, in dit geval 4 routers en 12 switches, gedefinieerd. 


\section{Implementatiefase}
In deze fase wordt het playbook getest. Eerst krijgen de netwerktoestellen een basisconfiguratie bestaande uit de nodige parameters voor een SSH-verbinding, 
alleen de nodige interfaces voor een SSH-verbinding en waar nodig routers naar de Ansible controller;
Deze stap kan niet overgeslaan worden, een SSH-verbinding zal in zowel een virtuele als fysieke omgeving opgezet moeten worden om de ansible controller te kunnen laten communiceren met de netwerktoestellen.
Zodra de nodige configuratie voor SSH-communicatie gebeurd is, worden de Ansible playbooks, die in de ontwerpfase geschreven zijn, uitgevoerd en getest. 


\section{Evaluatiefase}
Tijdens de evaulatiefase wordt gekeken of de werkelijke resultaten overeenkomen met de verwachtingen:
\begin{itemize}
    \item Is configuratie aan de hand van Ansible \textbf{sneller} dan manuele configuratie?
    \item Is deze manier van configureren \textbf{reproduceerbaar en consistent}? Met andere woorden kan je de playbooks meermaals heruitvoeren zonder afwijkingen in de eindconfiguratie?
    \item Is playbooks uitvoeren op één site even makkelijk en snel als de playbooks op meerdere sites uitvoeren?
    \item Worden \textbf{minder fouten}, die connectiviteit zouden verbreken, gemaakt bij configuratie met behulp van Ansible?
\end{itemize}