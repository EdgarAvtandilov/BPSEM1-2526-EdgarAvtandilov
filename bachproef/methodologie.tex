%%=============================================================================
%% Methodologie
%%=============================================================================

\chapter{\IfLanguageName{dutch}{Methodologie}{Methodology}}%
\label{ch:methodologie}

%% TODO: In dit hoofstuk geef je een korte toelichting over hoe je te werk bent
%% gegaan. Verdeel je onderzoek in grote fasen, en licht in elke fase toe wat
%% de doelstelling was, welke deliverables daar uit gekomen zijn, en welke
%% onderzoeksmethoden je daarbij toegepast hebt. Verantwoord waarom je
%% op deze manier te werk gegaan bent.
%% 
%% Voorbeelden van zulke fasen zijn: literatuurstudie, opstellen van een
%% requirements-analyse, opstellen long-list (bij vergelijkende studie),
%% selectie van geschikte tools (bij vergelijkende studie, "short-list"),
%% opzetten testopstelling/PoC, uitvoeren testen en verzamelen
%% van resultaten, analyse van resultaten, ...
%%
%% !!!!! LET OP !!!!!
%%
%% Het is uitdrukkelijk NIET de bedoeling dat je het grootste deel van de corpus
%% van je bachelorproef in dit hoofstuk verwerkt! Dit hoofdstuk is eerder een
%% kort overzicht van je plan van aanpak.
%%
%% Maak voor elke fase (behalve het literatuuronderzoek) een NIEUW HOOFDSTUK aan
%% en geef het een gepaste titel.

\section{Voorbereidingsfase}
In de voorbereidingsfase wordt onderzocht hoe een gemiddelde netwerktopologie van een kmo eruit ziet om een realistische en representatieve testomgeving op te zetten voor de proof-of-concept.
Hierbij wordt geen exacte kopie gemaakt van de netwerktopologie van een specifieke kmo, maar wordt er gefocust op het opstellen van een generiek model dat de meest voorkomende kenmerken van kmo-netwerken weergeeft.
Specifiek wordt er gekeken naar hoeveel toestellen, welk type toestellen en welke configuratie de toestellen moeten hebben.
Om het type toestellen te bepalen wordt er voornamelijk gekeken naar bestaande literatuur om te zien welk type toestellen het grootste marktaandeel hebben, dit vergroot de representativiteit van de testomgeving en de toepasbaarheid van de resultaten. 
Daarbij wordt ook rekening gehouden met configuratieprofielen en beheeropties, zodat de gekozen apparaten zoveel mogelijk aansluiten bij de praktijk van kmo's.
Deze fase toont dat Cisco, voor zowel switches als routers, het grootste marktaandeel heeft. Daarom werd er gekozen om voor de proof-of-concept Cisco toestellen te gebruiken in de testomgeving.

\section{Ontwerpfase}
In de ontwerpfase worden de testomgeving en Ansible controller effectief opgezet met de nodige inventarisbestanden en playbooks. Hiervoor wordt een virtueel netwerk opgezet in GNS3.
Het doel van deze fase is het definiëren van een duidelijke en reproduceerbare netwerktopologie die representatief is voor een typische kmo-omgeving en die voldoende complexiteit bevat om de voordelen van netwerkautomatisatie te kunnen aantonen.
Er werd ervoor gekozen om een bedrijfsomgeving te simuleren met meerdere sites. Deze keuze werd gemaakt omdat een kleine topologie geen goed beeld zou geven van de schaalbaarheid en tijdsbesparende kwaliteiten van automatisatie met Ansible.


Bij het ontwerp van de netwerktopologie wordt rekening gehouden met gangbare architecturale principes binnen bedrijfsnetwerken.
Zo wordt een duidelijke scheiding voorzien tussen verschillende netwerksegmenten, onder meer via VLAN-segmentatie, om zowel beheersbaarheid als veiligheid te verhogen.
Meerdere sites geven ook de mogelijkheid representatieve configuratie, zoals routing, Open Shortest Path First (OSPF) en Access Control Lists (ACL) te implementeren. Dit zijn elementen die op een uitgebreider netwerk goed aangetoond kunnen worden. 
Daarnaast wordt het netwerk zo opgebouwd dat uitbreidingen eenvoudig mogelijk blijven, wat aansluit bij de schaalbaarheidsvereisten van kmo's.

In het inventarisbestand worden alle routers en switches, in dit geval 4 routers en 12 switches, gedefinieerd. 
De ontworpen topologie omvat één centrale router die instaat voor de verbinding met externe netwerken, drie siterouters, één uplink switch per site en drie extra switches per site die fungeren als toegangspunt voor eindgebruikers en netwerkdiensten.
De switches zijn logisch gegroepeerd en ondersteunen meerdere VLAN's, zodat verschillende soorten verkeer van elkaar gescheiden kunnen worden.
Deze structuur laat toe om zowel basisconfiguraties als meer geavanceerde configuratiewijzigingen te automatiseren met behulp van Ansible.

Figuur~\ref{fig:topologie} geeft een visuele weergave van de ontworpen netwerktopologie die in de testomgeving werd gebruikt.
Dit schema vormt de referentie voor alle verdere configuratiestappen en automatisaties die in het kader van deze bachelorproef worden uitgevoerd.

\begin{figure}[H]
  \centering
  \includegraphics[width=0.85\textwidth]{figuren/topologie.png}
  \caption{Netwerktopologie voor de testomgeving}
  \label{fig:topologie}
\end{figure}

\section{Implementatiefase}
In deze fase worden de playbooks uitgevoerd en getest. 
Aangezien Ansible verbinding met de nodes maakt via SSH moet dit eerst opgezet worden op alle te beheren toestellen. De eerste SSH setup gebruikt placeholders voor enkele variabelen, zoals de hostname en domeinnaam, om tijd te besparen zodat de netwerkbeheerder dezelfde commando's op elk toestel kan uitvoeren zonder tijd te verliezen door de juiste variabelen mee te geven. 
Deze eerste basis SSH setup kan op deze manier gedaan worden, omdat deze tijdelijke SSH configuratie bij de automatisatie herschreven wordt met de gewenste variabelen.
Verder moeten de nodige interfaces voor een SSH-verbinding (dit is de interface die logisch, niet fysiek het dichts bij de Ansible controller ligt) en waar nodig routes naar de Ansible controller;
Deze stap kan niet overgeslaan worden, een SSH-verbinding zal in zowel een virtuele als fysieke omgeving opgezet moeten worden om de ansible controller te kunnen laten communiceren met de netwerktoestellen.

\vspace{1em}
\subsection{Basisconfiguratie om Ansible te laten communiceren met nodes}
Voor de beschreven basisconfiguratie moeten volgende commando's handmatig uitgevoerd worden op de routers en switches:
    \subsubsection{Opzetten SSH-verbinding}
    \begin{listing}[H]
    \caption{Basisconfiguratie voor SSH-communicatie op Cisco IOS}
    \label{lst:ssh-basisconfig}
    \begin{minted}{text}
    configure terminal
    hostname toestel
    ip domain-name dns.local
    no crypto key generate rsa
    crypto key zeroize rsa
    yes
    ip ssh version 2
    crypto key generate rsa modulus 2048
    username <gekozen username> privilege 15 secret <gekozen wachtwoord>
    line vty 0 4
    login local
    transport input ssh
    end
    write memory
    \end{minted}
    \end{listing}

    \subsubsection{Interface voor SSH-verbinding voorbereiden}
    \subsubsection{Routers}
    \begin{listing}[H]
    \caption{Basis interfaceconfiguratie op Cisco IOS routers}
    \label{lst:router-interface-basisconfig}
    \begin{minted}{text}    
    configure terminal
    interface <interface die het dichtste bij de ansible controller staat> 
    ip address <ip adres> <subnet mask>
    no shutdown
    \end{verbatim}
    \subsubsection{Switches}
    \begin{verbatim}
    enable
    configure terminal
    interface vlan99
    ip address <ip adres> <subnet mask>
    no shutdown
    interface <interface verbonden met siterouter/uplink switch>
    switchport access vlan 99
    no shutdown
    ip route 0.0.0.0 0.0.0.0 <next hop dichts naar Ansible controller>
    end
    \end{minted}
    \end{listing}

    \subsubsection{Routes naar Ansible controller}
    \begin{listing}[H]
    \caption{Routes naar Ansible controller}
    \label{lst:route-ansiblecontroller}
    \begin{minted}{text}   
        ip route 0.0.0.0 0.0.0.0 <next hop richting Ansible controller>
    \end{minted}
    \end{listing}

    \subsubsection{Routes van Ansible controller naar node netwerken}
    Deze stap moet per netwerk dat de netwerkbeheerder wil bereiken uitgevoerd worden.
    \begin{listing}[H]
    \caption{Routes naar te beheren netwerken}
    \label{lst:route-networks}
    \begin{minted}{text}   
        sudo route add -net <formaat netwerk/subnetmask in decimalen> <ip adres van de rechtstreeks met de Ansible controller verbonden interface>
    \end{minted}
    \end{listing}
Zodra er routes van de Ansible controller naar de nodes zijn, de netwerktoestellen zijn voorzien van een IP-adres waarop ze bereikt kunnen worden en SSH opgezet is, kan de netwerkbeheerder het playbook uitvoeren.

\subsection{Playbooks}
\subsubsection{Structuur van de Ansible-omgeving}
Voor de implementatie van de netwerkautomatisatie werd een gestructureerde Ansible-omgeving opgezet. De nadruk ligt hierbij op modulariteit, herbruikbaarheid en onderhoudbaarheid.
De gebruikte mappenstructuur volgt Ansible best practices. Hier wordt een onderscheid gemaakt tussen bestanden voor inventarisatie, variabelebeheer en configuratielogica.
Zo is er een inventarisbestand waarin alle netwerktoestellen gedefinieerd worden, een unieke map voor zowel groepsvariabelen als toestelvariabelen en tot slot een rollen map waarin specifieke configuratietaken staan.
Deze 'roles' vormen samen het uit te voeren playbook.
De structuur van de Ansible-omgeving staat hieronder weergegeven.

\vspace{3em}

\begin{verbatim}
ansible/
├── group_vars/
│   ├── all.yml
│   ├── edge_switches.yml
│   ├── routers.yml
│   └── switches.yml
├── host_vars/
│   ├── AntwerpenRouter.yml
│   ├── AntwerpenUplink.yml
│   ├── BrusselRouter.yml
│   ├── BrusselUplink.yml
│   ├── CompanyRouter.yml
│   ├── GentRouter.yml
│   └── GentUplink.yml
├── roles/
│   ├── base/
│   │   └── tasks/
│   │       └── main.yml
│   ├── interfaces/
│   │   └── tasks/
│   │       └── main.yml
│   ├── nat/
│   │   └── tasks/
│   │       └── main.yml
│   ├── ospf/
│   │   ├── defaults/
│   │   │   └── main.yml
│   │   └── tasks/
│   │       └── main.yml
│   ├── routes/
│   │   └── tasks/
│   │       └── main.yml
│   └── switchports/
│       └── tasks/
│           └── main.yml
├── ansible.cfg
├── hosts.yml
├── playbook.yml
└── requirements.yml
\end{verbatim}

\subsubsection{Inventarisatie}
De inventory wordt beheerd via een YAML-bestand (hosts.yml) waarin de netwerktoestellen logisch gegroepeerd zijn.
Alle netwerkwertoestellen staan onder de geneste groep 'all' met subgroep 'routers' en de geneste groep 'switches'.
De subgroep 'switches' bestaat uit de subgroepen 'uplink\_switches' en 'edge\_switches'.

Het inventarisbestand heeft de volgende structuur en kan in zijn volledigheid online geraadpleegd worden via GitHub

(\url{https://github.com/EdgarAvtandilov/BPSEM1-2526-EdgarAvtandilov/blob/main/code/ansible/hosts.yml}).
\subsubsection{Routes naar Ansible controller}
\begin{listing}[H]
\caption{Structuur inventarisatie}
\label{lst:inventarisatie-structuur}
\begin{minted}{text} 
all:
  children:
    routers:
      hosts:
        <router namen>
    switches:
      children:
        uplink_switches:
          hosts:
            <uplink switch namen>
        edge_switches:
          hosts:
            <edge switch namen>
\end{minted}
\end{listing}

\subsubsection{Variabelenbeheer}
Er wordt gebruik gemaakt van group\_vars en host\_vars om configuratieparameters respectievelijk op groeps- en toestelniveau te definiëren.
Dit laat toe om gemeenschappelijke instellingen te centraliseren, terwijl toestelspecifieke afwijkingen behouden blijven.

\subsubsection{Configuratielogica}
De configuratielogica is opgesplitst in afzonderlijke Ansible roles, waarbij elke role instaat voor een afgebakend onderdeel van de netwerkconfiguratie, zoals basisconfiguratie, interface-instellingen, routing, NAT en OSPF.
Deze modulaire aanpak verhoogt de leesbaarheid van de configuraties en maakt hergebruik en uitbreiding eenvoudig mogelijk.

Tot slot wordt een centraal playbook gebruikt om de verschillende roles in de juiste volgorde toe te passen op de gedefinieerde netwerktoestellen.
Deze aanpak sluit aan bij het Infrastructure as Code-principe en zorgt voor een reproduceerbare en consistente configuratie van de volledige testomgeving.

    
\section{Evaluatiefase}
Tijdens de evaluatiefase wordt gekeken of de werkelijke resultaten overeenkomen met de verwachtingen:
\begin{itemize}
    \item Is configuratie aan de hand van Ansible \textbf{sneller} dan manuele configuratie?
    \item Is deze manier van configureren \textbf{reproduceerbaar en consistent}? Met andere woorden kan je de playbooks meermaals heruitvoeren zonder afwijkingen in de eindconfiguratie?
    \item Is playbooks uitvoeren op één site even makkelijk en snel als de playbooks op meerdere sites uitvoeren?
    \item Worden \textbf{minder fouten}, die connectiviteit zouden verbreken, gemaakt bij configuratie met behulp van Ansible?
\end{itemize}

\subsection{Snelheid}
Netwerkconfiguratie met behulp van Ansible zorgt voor een snellere totaalconfiguratie van netwerktoestellen.
Om de tijdsbesparing te meten kregen 5 willekeurige switches en 5 willekeurige routers dezelfde handmatige en geautomatiseerde configuratie. De tijd per configuratie wordt gemeten en er werd een gemiddelde tijd berekend.
De tijd werd gemeten door de timer te starten nadat de toestellen volledig geboot waren en voor het eerste commando werd uitgevoerd. De timer werd gestopt zodra het laatste commando werd uitgevoerd. Wachttijden werden meegetimed, maar het schrijven van de running configuration naar de startup configuration werd niet meegerekend.
De gemeten tijden voor één enkel toestel zien eruit als volgt:
\begin{itemize}
    \item handmatige routerconfiguratie: 10 minuten 48 seconden
    \item handmatige switchconfiguratie: 13 minuten en 42 seconden
    \item geautomatiseerde routerconfiguratie: 8 minuten 
    \item geautomatiseerde switchconfiguratie: 9 minuten en 16 seconden
\end{itemize}

\subsubsection{Routers}
Zoals de bovenvermelde cijfers tonen duurt één router handmatig configureren gemiddeld 10 minuten en 48 seconden. Één router aan de hand van Ansible configureren daarentegen kostte gemiddeld 8 minuten.
Hoewel dit voor de configuratie van één toestel maar een tijdsbesparing is van 25,92\%, maken de tijdsbesparende kwaliteiten van automatische configuratie zich kenbaar in het configureren van meerdere toestellen.
Met een gemiddelde handmatige configuratieduur van 14 minuten en 13 seconden, zou alle routers in de opstelling configureren gemiddeld 43 minuten en 10 seconden in beslag nemen. 
Doordat Ansible toestellen parallel configureert, en niet één voor één zoals een netwerkbeheerder handmatig zou moeten doen, zie je de tijd die nodig is om de routers in de testopstelling te configureren drastisch verminderen. 
Zo duurt de geautomatiseerde configuratie van alle routers in de testopstelling 12 minuten en 51 seconden, een 70,23\% tijdsbesparing vergeleken met handmatige configuratie.
\subsubsection{Switches}
Voor de switches werden alleen de uplink switches getimed, aangezien zij wel een uitgebreide configuratie hebben en dus meer representatief zijn.
Ook bij de switches bespaart configuratie met behulp van Ansible tijd. Zo duurt één switch configureren gemiddeld 32,36\% minder lang. Een kleine, maar toch voelbare tijdbesparing.
Echter is de grootste tijdbesparing, net zoals bij routers, merkbaar bij het configureren van meerdere switches. 
De drie uplink switches handmatig configureren zou gemiddeld 41 minuten en 6 seconden in beslag nemen. 
Ansible maakt de configuratie af in 16 minuten en 52 seconden, een tijdsbesparing van 58,96\%.

\subsection{Reproduceerbaarheid en consistentie}
Dit onderzoek toont aan dat Ansible zorgt voor betere reproduceerbaarheid en consistentie.
Het gebruik van modules, die bij ontwerp idempotent zijn, zorgt ervoor dat de playbooks meermaals herhaald kunnen worden op de netwerktoestellen.
Bij het herhalen van de playbooks kijken de modules naar de, op Cisco genoemde, running configuration en wordt alleen afwijkende configuratie aangepast.
Bestaande configuratie die al overeenkomt met de gewenste configuratie wordt als 'ok' gemarkeerd door Ansible en waarna de volgende stap van het playbook begint.
Enkel afwijkende configuratie wordt dus aangepast en wordt als 'changed' gemarkeerd.
Hierdoor kan een netwerkbeheerder zien en volgen welke configuratie daadwerkelijk aangepast werd en daarmee ook bevestigen of gewenste aanpassingen effectief zijn uitgevoerd.

\subsection{Schaalbaarheid}
Naast tijdsbesparing en consistentie speelt ook schaalbaarheid een belangrijke rol bij netwerkbeheer binnen een kmo-context.
Bij handmatige configuratie neemt de benodigde configuratietijd lineair toe naarmate het aantal netwerktoestellen stijgt, aangezien elk toestel afzonderlijk moet worden geconfigureerd.
Dit vormt een belangrijke beperking bij groeiende netwerkomgevingen.
De evaluatie toont aan dat Ansible een schaalbare aanpak biedt voor netwerkconfiguratie.
Dankzij het gebruik van inventories en groepsgebaseerde configuraties kunnen configuratiewijzigingen in één keer worden toegepast op meerdere netwerktoestellen.
Wanneer extra routers of switches aan de inventaris worden toegevoegd, kunnen bestaande playbooks zonder aanpassing opnieuw worden gebruikt.
Het werk dat nodig is om het netwerk uit te breiden blijft daardoor beperkt, terwijl de configuratie consistent blijft over alle toestellen.
Daarnaast draagt de modulaire opbouw van de configuratie, via het gebruik van roles, bij aan de schaalbaarheid.
Nieuwe functionaliteiten of configuratieonderdelen kunnen worden toegevoegd door extra roles te definiëren of bestaande roles uit te breiden, zonder impact op reeds geïmplementeerde configuraties.
Deze aanpak maakt het mogelijk om het netwerk stapsgewijs uit te breiden en aan te passen aan veranderende eisen, zonder dat de complexiteit toeneemt.
Hieruit blijkt dat netwerkautomatisatie met Ansible niet alleen geschikt is voor kleine opstellingen, maar ook een schaalbare oplossing vormt voor kmo's met groeiende netwerkinfrastructuur.

\subsection{Foutvermindering}
Tot slot draagt automatisatie met Ansible naast snelheid, reproduceerbaarheid en schaalbaarheid bij aan het verminderen van configuratiefouten.
Bij handmatige netwerkconfiguratie worden toestellen één voor één via de command line interface geconfigureerd.
Dit proces is foutgevoelig, aangezien kleine typfouten, vergeten commando's of inconsistent toegepaste configuraties kunnen leiden tot netwerkstoringen of beveiligingsproblemen.
De evaluatie toont aan dat Ansible deze risico's aanzienlijk reduceert.
Configuraties worden vastgelegd in playbooks en roles, waardoor dezelfde configuratie telkens op identieke wijze wordt toegepast. Dit elimineert vergeten commando's en inconsistente configuraties.
Doordat configuraties niet manueel worden ingegeven, wordt het aantal typfouten ook aanzienlijk vermindert.
\vspace{1em}

Daarnaast zorgt het gebruik van gestructureerde variabelen en gestandaardiseerde modules voor consistente configuraties.
Een belangrijk mechanisme hierbij is de idempotente werking van Ansible-modules, deze controleren eerst de huidige configuratiestatus van een toestel alvorens wijzigingen toe te passen.
Hierdoor wordt voorkomen dat bestaande, correct werkende configuraties onbedoeld worden overschreven.
\vspace{1em}

Tijdens de uitvoering van een playbook wordt ook duidelijk gerapporteerd welke taken effectief wijzigingen hebben doorgevoerd.
Dit maakt het eenvoudiger om fouten te detecteren, te analyseren en indien nodig snel te corrigeren.
In tegenstelling tot manuele configuratie is er bovendien steeds een duidelijk overzicht van de gewenste configuratiestatus, wat troubleshooting vereenvoudigt.
\vspace{1em}

Op basis van deze bevindingen kan worden geconcludeerd dat Ansible niet alleen efficiënter is dan manuele configuratie, maar ook bijdraagt aan het verminderen van menselijke fouten en het verhogen van de stabiliteit van de netwerkinfrastructuur.