%%=============================================================================
%% Methodologie
%%=============================================================================

\chapter{\IfLanguageName{dutch}{Methodologie}{Methodology}}%
\label{ch:methodologie}

%% TODO: In dit hoofstuk geef je een korte toelichting over hoe je te werk bent
%% gegaan. Verdeel je onderzoek in grote fasen, en licht in elke fase toe wat
%% de doelstelling was, welke deliverables daar uit gekomen zijn, en welke
%% onderzoeksmethoden je daarbij toegepast hebt. Verantwoord waarom je
%% op deze manier te werk gegaan bent.
%% 
%% Voorbeelden van zulke fasen zijn: literatuurstudie, opstellen van een
%% requirements-analyse, opstellen long-list (bij vergelijkende studie),
%% selectie van geschikte tools (bij vergelijkende studie, "short-list"),
%% opzetten testopstelling/PoC, uitvoeren testen en verzamelen
%% van resultaten, analyse van resultaten, ...
%%
%% !!!!! LET OP !!!!!
%%
%% Het is uitdrukkelijk NIET de bedoeling dat je het grootste deel van de corpus
%% van je bachelorproef in dit hoofstuk verwerkt! Dit hoofdstuk is eerder een
%% kort overzicht van je plan van aanpak.
%%
%% Maak voor elke fase (behalve het literatuuronderzoek) een NIEUW HOOFDSTUK aan
%% en geef het een gepaste titel.

\section{Voorbereidingsfase}
In deze fase wordt onderzocht hoe een gemiddelde netwerktopologie van een kmo eruit ziet om een representatieve testomgeving op te zetten.
Specifiek wordt er gekeken naar hoeveel toestellen, welk type toestellen en welke configuratie de toestellen moeten hebben.
Om het type toestellen te bepalen wordt er voornamelijk gekeken naar bestaande literatuur om te zien welk type toestellen het grootste marktaandeel hebben, dit vergroot de representativiteit van de testomgeving en de toepasbaarheid van de resultaten. 
Daarbij wordt ook rekening gehouden met configuratieprofielen en beheeropties, zodat de gekozen apparaten zoveel mogelijk aansluiten bij de praktijk van kmo's.
Deze fase toont dat Cisco, voor zowel switches als routers, het grootste marktaandeel heeft. Daarom werd er gekozen om voor de proof-of-concept Cisco toestellen te gebruiken in de testomgeving.


\section{Ontwerpfase}
In de ontwerpfase worden de testomgeving en Ansible controller effectief opgezet.


\section{Implementatiefase}



\section{Evaluatiefase}