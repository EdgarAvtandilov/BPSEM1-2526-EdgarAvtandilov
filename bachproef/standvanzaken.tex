\chapter{\IfLanguageName{dutch}{Stand van zaken}{State of the art}}%
\label{ch:stand-van-zaken}

\section{De nood aan netwerkautomatisatie}
\label{sec:automatisatie}

Constante beschikbaarheid is de dag van vandaag zeer belangrijk in een bedrijfscontext.
Niet beschikbaar zijn, is voor bedrijven meer dan een kwestie van hun website die niet beschikbaar is. 
Downtime brengt ook grote kosten met zich mee. Zo toont een onderzoek van het Uptime Instute aan dat downtime voor 54\% van respondenten van hun enquête kosten van meer dan \$100.000 met zich mee brengt, en voor 1 in 5 respondenten liep dit op naar meer dan \$1.000.000 \autocite{UptimeInstitute2025}.
Dit onderzoek toont ook dat netwerk- en/of connectiviteitproblemen de oorzaak zijn van 34\% van IT-gerelateerde downtime. Een netwerk met 20 netwerktoestellen configureren vraagt om honderden configuratieregels, wat op zijn beurt honderden kansen op voor een menselijke fout biedt. 
Netwerken worden traditioneel via de command line interface (CLI) geconfigureerd.
Hoewel dit volledige controle geeft, toont het onderzoek van Uptime Institute aan dat menselijke fouten bijdragen tot 40\% van fouten die downtime veroorzaken. Menselijke interactie minimaliseren bij netwerkconfiguratie zou dus een perfecte oplossing kunnen bieden. 
Automatisatie biedt een antwoord op deze problemen door configuraties te centraliseren, sneller, reproduceerbaar en consistent, schaalbaar en foutloos te maken \autocite{Wagbrant2022}. 

